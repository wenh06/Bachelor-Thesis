\chapter{自守表示}
\label{chap:aut_rep}

本章主要介绍的是$GL(n)$的自守表示,因为``自守表示及其$L$-函数是Langlands纲领的核心''\cite{gelbart1984elementary}。本章最终的目的是给出守表示(automorphic representation)以及自守尖点表示(automorphic cuspidal representation)的定义,以及两个大定理Tensor product theorem和Multiplicity one theorem。

\section{一些基本概念}
\label{sec:aut_rep_intro}

我们先来介绍一些基本概念。

设$k$为固定的代数闭域。定义$k$上的$n$维仿射空间\cite{alg_geo}
$$
\mathbb{A}_k^n = \left\{ (a_1,\cdots,a_n) \  \middle| \  a_i\in k, 1 \leqslant i \leqslant n \right\},
$$
赋予$\mathbb{A}_k^n$ Zariski拓扑:取所有予$\mathbb{A}_k^n$所有代数集合的补为予$\mathbb{A}_k^n$的开集。这里,予$\mathbb{A}_k^n$的代数集合被定义为某$n$个变量的$k$系数多项式集合的公共零点集。

\begin{definition}
$\mathbb{A}_k^n$的一个不可约闭子集$X$就称为仿射代数簇。
\end{definition}
这里不可约指的是不能分解成两个真闭子集的并。设$R$为包含$k$的一个环,我们记$X(R)$为坐标取值在$R$中的$X$的点。即设$X = Z(T)$,为多项式集$T$的零点集,那么
$$
X(R) = \left\{ (a_1,\cdots,a_n) \in R^n \  \middle| \ \forall f\in T, f(a_i\in k, 1 \leqslant i \leqslant n) = 0 \right\}.
$$

接下来,我们给出域k上仿射代数群的定义。
\begin{definition}
设$G$为$k$上的一个仿射代数簇,被赋予了一个特殊点$1 \in G(k)$,群的乘法$G \times G \rightarrow G$由多项式给出,使得对任意的包含$k$的环$R$,$G(R)$在这个乘法下成为一个群。那么我们称$G$为域$k$上的一个仿射代数群。
\end{definition}

以下设$F$为一个整体域,即数域或函数域。记$\mathfrak{o}_v$为$F$在$v$处完备化$F_{v}$的整数环。定义$F$的阿代尔环$\mathbb{A}_F$和伊代尔群$\mathbb{A}_F^\times$如下:
\begin{align*}
\mathbb{A}_F & = \left\{ (a_v) \in \prod\limits_v F_v \ \middle|\ \text{对$F$的几乎所有的有限素点$v$有} a_v\in \mathfrak{o}_v \right\}, \\[1em]
\mathbb{A}_F^\times & = \left\{ (a_v) \in \prod\limits_v F_v \ \middle|\ \text{对$F$的几乎所有的有限素点$v$有} a_v\in \mathfrak{o}_v^\times \right\}.
\end{align*}

我们给出表示的相关概念。

\begin{definition}
设$G$为任一群,$V$为某个域$k$上的线性空间。如果存在群同态$\rho: G \rightarrow \operatorname{GL}(V)$,其中$\operatorname{GL}(V)$为一般线性群,那么我们称$(\rho, V)$,简记成$V$或者$\rho$。
\end{definition}
一个等价的定义是$G$在$V$上有一个线性作用,则称$V$为$G$的一个表示。另一个等价的定义是$V$是具有$kG$-模结构,则称$V$为$G$的一个表示。

若$U$是$V$的一个子空间, 且在$G$的作用下为不变子空间, 那么称$(\rho|_U,U)$为$(\rho, V)$的一个子表示。考虑商空间$V/U$,对$g \in G, x + U \in V/U$,定义$g(x + U) = g(x) + U$,那么我们得到$G$在$V/U$上的一个作用,称为$(\rho, V)$的一个商表示,记为$(V/U, \rho_{V/U})$。群$G$的非零表示$(\rho, V)$称为是不可约的,如果他没有非平凡的子表示。

设$(\rho_1, V_1)$和$(\rho_2, V_2)$是群$G$的两个表示,我们称他们是同构的,如果存在单满的$k$-线性映射$f: V_1 \rightarrow V_2$,使得对任一$g \in G$有下面的交换图
\begin{center}
\begin{tikzcd}
V_1 \arrow[r, "f"] \arrow[d, "\rho_1(g)"'] & V_2 \arrow[d, "\rho_2(g)"] \\
V_1 \arrow[r, "f"] & V_2
\end{tikzcd}
\end{center}

关于表示的进一步知识,可以参考\inlinecite{gp_rep}。

现在再来介绍限制直积\cite{nt1}和限制张量积\cite{bump1998automorphic}的概念。

\begin{definition}
设$(G_\lambda)_{\lambda\in\Lambda}$为一族局部紧群,$S$为$\Lambda$的有限子集合,对于每个$\lambda \in \Lambda \setminus S$给定$G_\lambda$的一个紧开子群$U_\lambda$。称直积群$\displaystyle \prod\limits_{\lambda\in\Lambda} G_\lambda$的子群
$$
\left\{ (x_\lambda)_{\lambda\in\Lambda} \in \prod\limits_{\lambda\in\Lambda} G_\lambda  \ \middle|\ \text{对几乎所有$\lambda \in \Lambda \setminus S$有$x_\lambda \in U_\lambda$} \right\}
$$
为$(G_\lambda)_{\lambda\in\Lambda}$关于$(U_\lambda)_{\lambda \in \Lambda \setminus S }$的限制直积。
\end{definition}

我们对限制直积给出如下拓扑。对于包含$S$的$\Lambda$的有限子集$T$,考虑$\prod\limits_{\lambda\in\Lambda} G_\lambda$的子集
$$
G(T) = \prod\limits_{\lambda\in T} G_\lambda \times \prod\limits_{\lambda\in \Lambda \setminus T} U_\lambda.
$$
于是
$$
\prod\limits_{\lambda\in\Lambda} G_\lambda = \bigcup\limits_T G(T).
$$
在每个$G(T)$中引入直积拓扑,对于$\prod\limits_{\lambda\in\Lambda} G_\lambda$的子集$V$,如果对所有$T$,$V \cap
G(T)$为$G(T)$中的开集,则定义$V$为开集。

以下简记$\mathbb{A}_F$为$A$。易知$A$是$F_v$关于$\mathfrak{o}_v$的限制直积。在限制直积拓扑下,$A$是局部紧群。记$A_f$为有限阿代尔组成的环,即
$$
A_f = \left\{ (a_\lambda)_{\lambda\in\Lambda} \in A \ \middle|\ a_v = 1, \text{对所有的无限素点}v \right\}.
$$
同样地,$GL(n, A)$以显然的方式视为$GL(n, F_v)$ 关于$GL(n, \mathfrak{o}_v)$的限制直积。
