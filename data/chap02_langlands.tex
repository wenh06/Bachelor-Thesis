\chapter{Langlands猜想}
\label{chap:langlands}
% finished, NOT checked,

\section{\texorpdfstring{$L$}{L}-群}
\label{sec:l_gp}
我们先依照\inlinecite{alg_gp,bump1998automorphic}给出仿射代数群更多的概念。

设$G$为域$F$上的一个代数群,那么$G$有唯一一个极大不可约子集包含单位元,记为$G^0$。

\begin{definition}
如果$G = G^0$,则称$G$是连通的。
\end{definition}
记n阶对角阵全体为
$$
D(n,F) = \left\{ \begin{pmatrix} \ast & & 0 \\ & \ddots & \\ 0 & & \ast\end{pmatrix} \right\}
$$
这是一个代数群,是$\operatorname{Mat}_{n\times n}$的子群。

\begin{definition}
设$G$为一个代数群,如果$G$与$D(n, F)$的一个子群同构,则称$G$是可对角化的。
\end{definition}

\begin{definition}
如果$G$是连通的的可对角化代数群,则称$G$是一个环面(torus)。
\end{definition}

\begin{definition}
一个仿射代数群$G$被称为是简约的(reductive),如果$G$没有非平凡的正规幂幺子群。$G$被称为半单(semisimple)的如果$G$是简约代数群,而且没有非平凡的正规环面。
\end{definition}
若$G$是一个简约代数群,我们定义$G$的一个Borel子群为$G$的一个极大的连通可解子群。例如
$$
T(n,F) = \left\{ \begin{pmatrix} \ast & \cdots & \ast \\ & \ddots & \vdots \\ 0 & & \ast \end{pmatrix} \right\}
$$
$n$阶上三角阵,就是域$F$上代数群$\operatorname{GL}(n)$的一个Borel子群。可以证明,$G$的所有Borel子群是共轭的。

\begin{definition}
设$T$为域$F$上的一个环面,称$T$在$F$上是分裂的(split),如果$T$同构于$n$个$\mathbb{G}_m$的直积。
\end{definition}
这里的$n$为某个正整数,$\mathbb{G}_m$为域$F$上的乘法群。

\begin{definition}
设$G$为域$F$上的一个简约代数群,称$G$是分裂的(split),如果$G$有一个在$F$上分裂的极大环面。
\end{definition}

设$F$为一个局部域或者整体域,$G$为$F$上的一个简约代数群。那么一个连通的$L$-群$\prescript{L}{}{G}^0$指的是一个典范地与$F$相关的复的李群。下表给出了一些例子

\begin{table}[H]
\centering
\begin{tabular}{p{0.17\textwidth}p{0.17\textwidth}}
\toprule
$G$  & $\prescript{L}{}{G}^0$ \\
\midrule
$\operatorname{GL}(n)$ & $\operatorname{GL}(n, \mathbb{C})$ \\
$\operatorname{SL}(n)$ & $\operatorname{SL}(n, \mathbb{C})$ \\
$\operatorname{PGL}(n)$ & $\operatorname{PGL}(n, \mathbb{C})$ \\
$\operatorname{Sp}(2n)$ & $\operatorname{SO}(2n+1, \mathbb{C})$ \\
$\operatorname{SO}(2n+1)$ & $\operatorname{Sp}(2n, \mathbb{C})$ \\
$\operatorname{SO}(2n)$ & $\operatorname{SO}(2n, \mathbb{C})$ \\
\bottomrule
\end{tabular}
\end{table}

我们举一个例子,来说明这种对应。

\begin{example}
设$F$是一个非阿局部域(即不是实数域或复数域),记$\mathfrak{o}$为其整数环,$v$为其赋值。$ \varpi$为$\mathfrak{o}$的唯一的极大理想的生成元。设$G$为$F$上的一个简约代数群,并且在$F$上分裂。设$\chi_1, \cdots , \chi_n$为$F^\times$的拟特征,取$\operatorname{GL}(n)$的Borel子群$B(F) = T(n, F)$。记$\chi$为$B(F)$的如下定义的拟特征:
$$
\chi \left( \begin{pmatrix} y_1 & \ast & \cdots & \ast \\ & y_1 & \cdots & \ast \\ & & \ddots & \vdots \\ & & & y_n \end{pmatrix} \right) := \chi_1(y_1) \cdots \chi_n(y_n),
$$
即我们给出了$B(F)$的一个一维表示。我们把这个表示诱导为$\operatorname{GL}(n, F)$上的表示,记为$\mathcal{B}(\chi_1, \cdots, \chi_n)$。我们有如下的定理
\end{example}

\begin{theorem}
若$\chi_1, \cdots , \chi_n$是非分歧的,那么表示$\mathcal{B}(\chi_1, \cdots, \chi_n)$包含$\operatorname{GL}(n, \mathfrak{o})$不动的元素。$\mathcal{B}(\chi_1, \cdots, \chi_n)$有唯一的一个合成因子,使得这个合成因子是一个非分歧表示。$\operatorname{GL}(n, F)$的每个非分歧表示都是这种形式。
\end{theorem}

$\chi_i (1 \leqslant i \leqslant n)$非分歧指的是$\chi_i(x) = 1$,对所有的$\lvert x \rvert_v = 1$。非分歧表示指的是这个表示在$\operatorname{GL}(n, F)$的极大紧子群$\operatorname{GL}(n, \mathfrak{o})$作用下不动的元素\inlinecite{gelbart1984elementary}。一般地,设$G$为$F$上的任一简约代数群,一个容许的不可约表示被称为不可约的,如果这个表示在$G(F)$的一个特殊极大紧子群作用下有不动的元素。在\inlinecite{bump1998automorphic}中,这样的表示被称为是Spherical的。

如果$\pi$是$\operatorname{GL}(n, F)$的一个非分歧的表示,根据上面这个定理,我们知道存在$F^\times$的非分歧特征$\chi_1, \cdots, \chi_n$,使得$\chi$是$\mathcal{B}(\chi_1, \cdots, \chi_n)$的一个合成因子。那么我们可以用
$$
\begin{pmatrix}
\chi_1(\varpi) & & \\
& \ddots & \\
& & \chi_n(\varpi)
\end{pmatrix} \in \prescript{L}{}{G}^0 = \operatorname{GL}(n, \mathbb{C})
$$
来参数化$\pi$由同构关系定义的等价类。

关于$L$-群的构造,参阅\inlinecite{bump1998automorphic},在\inlinecite{alg_gp}中也有提及, 更严格更详细构造在Borel的\inlinecite{borel1979automorphic}。总之,我们可以得到以下的一些结论\inlinecite{gelbart1984elementary}:

设$G$为$F$上的一个简约代数群。取定$F$的一个足够大的有限Galois扩张$\Omega$,特别地,$G$在$\Omega$上分裂。我们可以定义一个复的简约李群$\prescript{L}{}{G}^0$,并给出$\operatorname{Gal}(\Omega/F)$在$\prescript{L}{}{G}^0$上的一个作用,使得半直积
$$
\prescript{L}{}{G} = \prescript{L}{}{G}^0 \rtimes \operatorname{Gal}(\Omega/F)
$$
满足如下条件:
\begin{itemize}
\item[(i)] 如果$G$在$F$上是分裂的,那么$\prescript{L}{}{G}$是$\prescript{L}{}{G}^0$和$\operatorname{Gal}(\Omega/F)$的直积,$\operatorname{Gal}(\Omega/F)$在$\prescript{L}{}{G}^0$上的作用是平凡的。
\item[(ii)] 一般地,如果$F$的素点$v$在
中非分歧,记$\operatorname{Frob}_v$为$v$对某个在他之上的素点的Frobenius自同构。$G(F_v)$的非分歧表示$\pi_v$和$\prescript{L}{}{G}$的半单共轭类$t(\pi_v)$有一个一一对应,并且$t(\pi_v)$在$\operatorname{Gal}(\Omega/F)$上的投影就是$\operatorname{Frob}_v$所在的共轭类。
\end{itemize}

我们把这个群$\prescript{L}{}{G}$称作是$G$的$L$-群。Frobenius自同构的定义见\inlinecite{nt1}或者\inlinecite{bump1998automorphic}。设$r : \prescript{L}{}{G} \rightarrow \operatorname{GL}_d(\mathbb{C})$为一个群同态,使得$r$在$\prescript{L}{}{G}^0$上的限制是复解析的,那么我们
称$r$为$\prescript{L}{}{G}$的一个表示。我们称一个连续的群同态$\rho: \prescript{L}{}{G} \rightarrow \prescript{L}{}{G}'$为一个$L$-同态,如果$\rho$与两个$L$-群分别在$\operatorname{Gal}(\Omega/F)$上的投影相容,即有如下交换图:

\begin{center}
\begin{tikzcd}
\prescript{L}{}{G} \ar[rr, "\rho"] \ar[rd] & & \prescript{L}{}{G}' \ar[ld] \\
& \operatorname{Gal}(\Omega/F)
\end{tikzcd}
\end{center}

\section{Langlands猜想}

现在,我们来叙述Langlands纲领的主要内容。

\begin{conjecture}[\inlinecite{gelbart1984elementary}]
\label{conj:reciprocity}
设$E$为$F$的一个有限Galois扩张,记$G = \operatorname{Gal}(E/F)$,并设$\sigma: G \rightarrow \operatorname{GL}(n, \mathbb{C})$为$G$的一个不可约表示。那么存在$\operatorname{GL}(n)$的一个自守尖点表示$\pi_\sigma$使得
$$
L(s, \pi_\sigma) = L(s, \sigma).
$$
\end{conjecture}

这个猜想可以被称为“互反律”猜想。但是它的逆命题是不成立的\inlinecite{bump1998automorphic}。下面这个猜想是关于由自守表示定义的L-函数的延拓性和函数方程的。

\begin{conjecture}[\inlinecite{gelbart1984elementary}]
设$\pi = \otimes \pi_v$是$G = \operatorname{GL}(n)$的一个自守表示,$r$是$\prescript{L}{}{G}$的一个有限维表示,那么
$$
L(s,\pi,r) = \prod\limits_{v\not\in S_\pi} \det \left[ I - r(t(\pi_v))Nv^{-s} \right]^{-1}
$$
可以亚纯地延拓到整个复平面$\mathbb{C}$,并且有$L(s, \pi, r) \leftrightarrow L(1 − s, \widetilde{\pi}, r)$的函数方程。
\end{conjecture}

这里的$S_\pi$为$F$的使得$\pi_v$非分歧的素点$v$的集合,而$\widetilde{\pi}$指的是$\pi$的反轭表示。关于
反轭表示,见\inlinecite{gp_rep}第一章3.2。

\begin{conjecture}[函子性猜想\inlinecite{gelbart1984elementary}]
\label{conj:functor}
设$G,G'$为简约代数群,$\rho: \prescript{L}{}{G} \rightarrow \prescript{L}{}{G}'$是一个$L$-同态。那么对于$G$的每个自守表示$\pi = \otimes \pi_v$,存在$G'$的一个自守表示$\pi' = \otimes \pi_v'$,使得对所有的非分歧的$v$,$t(\pi_v')$是$\prescript{L}{}{G}'$包含$t(\pi_v')$的共轭类。并且对$\prescript{L}{}{G}'$的任意有限维表示$r'$有
$$
L(s, \pi', r') = L(s, \pi, r'\circ\rho).
$$
\end{conjecture}

\section{一些简单的例子}
\label{sec:eg}
最后,我们举几个简单的例子,看看Langlands纲领如何导出一些经典的结果。

\begin{example}
当$n = 1$,$E/F$是一个Abel扩张时,记$G = \operatorname{Gal}(E/F)$,$\sigma: G \rightarrow \mathbb{C}^\times$是$G$的一个特征,猜想\ref{conj:reciprocity}很直接地即是下面的定理
\end{example}

\begin{theorem}[Artin]
设$E/F$为一个Abel扩张,$\sigma: G \rightarrow \mathbb{C}^\times$是$G$的一个特征,那么存在$E$的Hecke特征$\chi$,使得
$$
L_{E/F}(s, \sigma) = L(s, \chi).
$$
\end{theorem}

这里的$L_{E/F}(s, \sigma)$为Artin $L$-函数,$L(s, \chi)$为Hecke $L$-函数,定义见\inlinecite{gelbart1984elementary}或\inlinecite{bump1998automorphic}。

猜想2.2退化为下面的定理

\begin{theorem}[Artin-Brauer]
$L(s, \chi)$可以亚纯地延拓至整个复平面$\mathbb{C}$,并且有函数方程
$$
L(s, \sigma) = \varepsilon(s, \sigma) L(1-s, \widetilde{\sigma}).
$$
\end{theorem}

\begin{example}
设$E/F$为数域的Galois扩张,$\pi$是$\operatorname{GL}(n)$在$F$上的一个自守表示,那么根据函子性猜想\ref{conj:functor},应该存在定义在$E$上的代数群$\operatorname{GL}(n)$的一个表示$\pi'$,典范地与$\pi$相关,被称为基变换提升(base change lift)。这是因为,我们可以分别取$G,G'$为定义在$F$以及$E$上的代数群$\operatorname{GL}(n)$,那么
\begin{gather*}
\prescript{L}{}{G} = \operatorname{GL}(n, \mathbb{C}) \times \operatorname{Gal}(\Omega/F), \\
\prescript{L}{}{G}' = \operatorname{GL}(n, \mathbb{C}) \times \operatorname{Gal}(\Omega/E).
\end{gather*}
那么典范的映射
$$
\rho: (\operatorname{GL}(n, \mathbb{C}) \times \operatorname{Gal}(\Omega/F) = ) \prescript{L}{}{G} \rightarrow \prescript{L}{}{G}' ( = \operatorname{GL}(n, \mathbb{C}) \times \operatorname{Gal}(\Omega/E))
$$
是一个$L$-同态,因此这样的$\pi'$应该是存在的。对$E/F$为循环扩张的情况,这样的提升的存在性已经得到证明\cite{bump1998automorphic}。
\end{example}
