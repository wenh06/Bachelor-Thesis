\chapter{Langlands猜想}
\label{chap:langlands}

\section{$L$-群}
\label{sec:l_gp}
我们先依照\inlinecite{alg_gp,bump1998automorphic}给出仿射代数群更多的概念。

设$G$为域$F$上的一个代数群,那么$G$有唯一一个极大不可约子集包含单位元,记为$G^0$。

\begin{definition}
如果$G = G^0$,则称$G$是连通的。
\end{definition}
记n阶对角阵全体为
$$
D(n,F) = \left\{ \begin{pmatrix} \ast & & 0 \\ & \ddots & \\ 0 & & \ast\end{pmatrix} \right\}
$$
这是一个代数群,是$\operatorname{Mat}_{n\times n}$的子群。

\begin{definition}
设$G$为一个代数群,如果$G$与$D(n, F)$的一个子群同构,则称$G$是可对角化的。
\end{definition}

\begin{definition}
如果$G$是连通的的可对角化代数群,则称$G$是一个环面(torus)。
\end{definition}

\begin{definition}
一个仿射代数群$G$被称为是简约的(reductive),如果$G$没有非平凡的正规幂幺子群。$G$被称为半单(semisimple)的如果$G$是简约代数群,而且没有非平凡的正规环面。
\end{definition}
若$G$是一个简约代数群,我们定义$G$的一个Borel子群为$G$的一个极大的连通可解子群。例如
$$
T(n,F) = \left\{ \begin{pmatrix} \ast & \cdots & \ast \\ & \ddots & \vdots \\ 0 & & \ast \end{pmatrix} \right\}
$$
$n$阶上三角阵,就是域$F$上代数群$\operatorname{GL}(n)$的一个Borel子群。可以证明,$G$的所有Borel子群是共轭的。

\begin{definition}
设$T$为域$F$上的一个环面,称$T$在$F$上是分裂的(split),如果$T$同构于$n$个$\mathbb{G}_m$的直积。
\end{definition}
这里的$n$为某个正整数,$\mathbb{G}_m$为域$F$上的乘法群。

\begin{definition}
设$G$为域$F$上的一个简约代数群,称$G$是分裂的(split),如果$G$有一个在$F$上分裂的极大环面。
\end{definition}

设$F$为一个局部域或者整体域,$G$为$F$上的一个简约代数群。那么一个连通的$L$-群$\prescript{L}{}{G}^0$指的是一个典范地与$F$相关的复的李群。下表给出了一些例子

\begin{table}[H]
\centering
\begin{tabular}{p{0.17\textwidth}p{0.17\textwidth}}
\toprule
$G$  & $\prescript{L}{}{G}^0$ \\
\midrule
$\operatorname{GL}(n)$ & $\operatorname{GL}(n, \mathbb{C})$ \\
$\operatorname{SL}(n)$ & $\operatorname{SL}(n, \mathbb{C})$ \\
$\operatorname{PGL}(n)$ & $\operatorname{PGL}(n, \mathbb{C})$ \\
$\operatorname{Sp}(2n)$ & $\operatorname{SO}(2n+1, \mathbb{C})$ \\
$\operatorname{SO}(2n+1)$ & $\operatorname{Sp}(2n, \mathbb{C})$ \\
$\operatorname{SO}(2n)$ & $\operatorname{SO}(2n, \mathbb{C})$ \\
\bottomrule
\end{tabular}
\end{table}

我们举一个例子,来说明这种对应。

\begin{example}
设$F$是一个非阿局部域(即不是实数域或复数域),记$\mathfrak{o}$为其整数环,$v$为其赋值。$ \varpi$为$\mathfrak{o}$的唯一的极大理想的生成元。设$G$为$F$上的一个简约代数群,并且在$F$上分裂。
\end{example}




\section{Langlands猜想}

现在,我们来叙述Langlands纲领的主要内容。
