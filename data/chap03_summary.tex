\chapter{总结}
\label{chap:summary}
% finished

正如Gelbart在\inlinecite{gelbart1984elementary}中所言:

``Here lies the agony as well as the ecstasy of Langlands’s Program. To merely state the conjectures correctly requires much of the machinery of class field theory, the structure theory of algebraic groups, the representation theory of real and p-adic groups, and (at least) the language of algebraic geometry. In other words, though the promised rewards are great, the initiation process is forbidding.''

数论是一门深奥而奇妙的学问,Langlands纲领更是体现了数论世界某种大范围的联系。要理解Langlands提出的这一系列猜想尚需要大量的知识准备,本文就是这样的一种尝试。一个遗憾之处在于,本文没有涉及最近出现的几何Langlands纲领,甚至还没有去探究Langlands本人阐述他自己想法的文章《Problems in the Theory of Automorphic forms》。

虽然本文比较粗浅,但更重要的是探索、学习的这个过程。从一开始的一点破碎的感觉,后来阅读各种书籍,这种感觉逐渐有了某种连贯性,现在终于能够窥见这神奇的世界的一角,我已经被其中简洁而深刻的美所震撼。希望自己今后能在这个问题上有所突破吧。
