\thusetup{
  %******************************
  % 注意:
  %   1. 配置里面不要出现空行
  %   2. 不需要的配置信息可以删除
  %******************************
  %
  %=====
  % 秘级
  %=====
%   secretlevel={公开},
%   secretyear={0},
  %
  %=========
  % 中文信息
  %=========
  ctitle={自守表示,Langlands 纲领},
  cdegree={理学学士},
  cdepartment={数学科学系},
  cmajor={数理基础科学专业},
  cauthor={文豪},
  csupervisor={印林生\hspace{1em}教授},
  cassosupervisor={}, % 副指导老师
  ccosupervisor={}, % 联合指导老师
  % 日期自动使用当前时间,若需指定按如下方式修改:
   cdate={2010年6月},
  %
  % 博士后专有部分
%   cfirstdiscipline={计算机科学与技术},
%   cseconddiscipline={系统结构},
%   postdoctordate={2009年7月——2011年7月},
%   id={编号}, % 可以留空: id={},
%   udc={UDC}, % 可以留空
%   catalognumber={分类号}, % 可以留空
  %
  %=========
  % 英文信息
  %=========
  etitle={Automorphic Representations, Langlands Program},
  % 这块比较复杂,需要分情况讨论:
  % 1. 学术型硕士
  %    edegree:必须为Master of Arts或Master of Science(注意大小写)
  %             “哲学、文学、历史学、法学、教育学、艺术学门类,公共管理学科
  %              填写Master of Arts,其它填写Master of Science”
  %    emajor:“获得一级学科授权的学科填写一级学科名称,其它填写二级学科名称”
  % 2. 专业型硕士
  %    edegree:“填写专业学位英文名称全称”
  %    emajor:“工程硕士填写工程领域,其它专业学位不填写此项”
  % 3. 学术型博士
  %    edegree:Doctor of Philosophy(注意大小写)
  %    emajor:“获得一级学科授权的学科填写一级学科名称,其它填写二级学科名称”
  % 4. 专业型博士
  %    edegree:“填写专业学位英文名称全称”
  %    emajor:不填写此项
%   edegree={Doctor of Engineering},
%   emajor={Computer Science and Technology},
%   eauthor={Xue Ruini},
%   esupervisor={Professor Zheng Weimin},
%   eassosupervisor={Chen Wenguang},
  % 日期自动生成,若需指定按如下方式修改:
  % edate={December, 2005}
  %
  % 关键词用“英文逗号”分割
  ckeywords={Langlands 纲领, 自守表示, $L$-群, $L$-函数},
  ekeywords={Langlands Program, Automorphic Representations, $L$-group, $L$-functions}
}

% 定义中英文摘要和关键字
\begin{cabstract}
  本文主要介绍了$GL(n)$上的自守表示理论,以及对Langlands纲领进行了一个初步的综述。参照有关的文献,给出了一些基本的结果和主要的猜想。
\end{cabstract}

% 如果习惯关键字跟在摘要文字后面,可以用直接命令来设置,如下:
% \ckeywords{\TeX, \LaTeX, CJK, 模板, 论文}

\begin{eabstract}
   This article mainly introduces the automorphic representation theory for $GL(n)$, and gives an elementary survey of the Langlands Program. Through consulting relevant references, this article presents some fundamental results and conjectures.
\end{eabstract}

% \ekeywords{\TeX, \LaTeX, CJK, template, thesis}
