
%%% Local Variables:
%%% mode: latex
%%% TeX-master: t
%%% End:
\secretlevel{绝密} \secretyear{2100}

\ctitle{自守表示,Langlands 纲领}
% 根据自己的情况选,不用这样复杂
\makeatletter
\ifthu@bachelor\relax\else
  \ifthu@doctor
    \cdegree{工学博士}
  \else
    \ifthu@master
      \cdegree{工学硕士}
    \fi
  \fi
\fi
\makeatother


\cdepartment[数学系]{数学科学系} \cmajor{数理基础科学专业}
\cauthor{文豪} \csupervisor{印林生\ 教授}
% 如果没有副指导老师或者联合指导老师,把下面两行相应的删除即可。
% 日期自动生成,如果你要自己写就改这个cdate
%\cdate{\CJKdigits{\the\year}年\CJKnumber{\the\month}月}


% 定义中英文摘要和关键字
\begin{cabstract}
  本文主要介绍了$GL(n)$上的自守表示理论,以及对Langlands纲领进行了一个初步的综述。参照有关的文献,
  给出了一些基本的结果和主要的猜想。
\end{cabstract}

\ckeywords{Langlands 纲领, 自守表示, $L$-群, $L$-函数}

\begin{eabstract}
   This article mainly introduces the automorphic representation
   theory for $GL(n)$, and gives an elementary survey of the
   Langlands Program. Through consulting relevant references, this
   article presents some fundamental results and conjectures.
\end{eabstract}

\ekeywords{Langlands Program, Automorphic representation, $L$-group,
$L$-functions}
